% vim: spl=pt
\documentclass[portuguese,minted]{artigo}
\usepackage{booktabs}
\usepackage{circuitikz}

% \DeclareUnicodeCharacter{3B3}{$\gamma$}
\setmonofont{Iosevka Nerd Font Mono}

\setminted{
    frame=lines,
    framesep=2mm, 
    baselinestretch=1.2,
    fontsize=\footnotesize,
    linenos
}

\author{Louis Bergamo Radial\\8992822}
\title{MAP0214 - Cálculo Numérico Aplicado à Física\\Exercício Programa IV - Equações Diferenciais Ordinárias}
\begin{document}
\maketitle

Para os métodos numéricos, a linguagem C foi utilizada com o compilador \verb|gcc|, enquanto que os gráficos produzidos foram feitos com a biblioteca \verb|matplotlib| do Python.

\section{Método de Runge-Kutta clássico de quarta ordem}
O Exercício Programa (EP) exige a solução de EDOs de segunda ordem \(\ddot{x}(t) = g(t, x(t), \dot{x}(t))\), que podem ser manipuladas para serem tratadas como um sistema de EDOs de primeira ordem segundo
\begin{equation*}
    \dot{x}(t) = v(t) \quad\text{e}\quad \dot{v}(t) = g(t, x(t), v(t)).
\end{equation*}

Para a implementação de integração de equações diferenciais ordinárias (EDOs) foi criada uma \emph{struct} \(\verb|ode|\) que armazena tanto o ponteiro da função que caracteriza a equação diferencial quanto os parâmetros utilizados nesta função. A função \verb|compute| calcula a função da EDO no instante desejado, já utilizando os seus parâmetros.
\inputminted[firstline=5,lastline=16]{c}{src/main.c}

Para testar a implementação do método de Runge-Kutta clássico de quarta ordem (RK4) e comparar com o método de Euler, consideramos a EDO \(\ddot{y} = \dot{y} + y - t^3 - 4t^2 + 4t + 2\) com \(y(0) = \dot{y}(0) = 0,\) cuja solução analítica é \(y = t^3 + t^2.\) Para ambos os métodos, uma subrotina, \verb|euler| e \verb|rk4|, foi criada que calcula um passo da iteração e uma subrotina, \verb|solve_euler| e \verb|solve_rk4|, que itera sobre todo o período de integração e escreve os dados em um arquivo. Para uso no resto do EP, a função \verb|solve_rk4| também permite que a parte dos dados que é escrita em um arquivo seja feita a partir de um instante intermediário do período de integração.
\inputminted[firstline=18,lastline=88]{c}{src/main.c}

Para a comparação, as interações foram feitas com o passo \(h = 0.01,\) e os resultados foram plotados na \cref{fig:ex1}. Para o método de Euler, ao final da integração foi obtido
\begin{equation*}
    y_{\mathrm{Euler}}(6) = 51.206789
    \quad\text{e}\quad
    \dot{y}_{\mathrm{Euler}}(6) = -204.564558,
\end{equation*}
enquanto que para RK4, foi obtido
\begin{equation*}
    y_{\mathrm{RK4}}(6) = 251.999983
    \quad\text{e}\quad
    \dot{y}_{\mathrm{RK4}}(6) = 119.999972,
\end{equation*}
que devem ser comparados com \(y(6) = 252\) e \(\dot{y}(6) = 120.\)

\begin{figure}[!ht]
    \centering
    \includegraphics[width=0.5\textwidth]{src/1.png}
    \caption{Comparação do método RK4 e de Euler}
    \label{fig:ex1}
\end{figure}

Na \cref{fig:ex1} não é possível discernir as curvas da solução analítica e a integrada numericamente por RK4, mesmo com o passo modesto de \(h = 0.01.\) A \cref{fig:ex1b} apresenta o erro relativo entre a solução analítica e RK4, tanto para \(y\) quanto para \(\dot{y},\) mostrando que os erros foram da ordem de \(10^{-8}\) para \(y\) e \(10^{-7}\) para \(\dot{y}.\) Apesar destes gráficos mostrarem que o erro aumenta conforme o intervalo de integração aumenta, utilizaremos um passo menor para as próximas tarefas, o que reduz o erro relativo.
\begin{figure}[!ht]
    \centering
    \includegraphics[width=0.5\textwidth]{src/1b.png}
    \caption{Comparação do método RK4 e da solução analítica}
    \label{fig:ex1b}
\end{figure}

\section{Potencial poço duplo}
Consideramos o potencial de poço duplo, com equação de movimento
\begin{equation*}
    \ddot{x} = \frac12 x(1 - x^2) - 2 \gamma \dot{x} + F \cos(\omega t),
\end{equation*}
onde os parâmetros do sistema são o amortecimento \(\gamma,\) a amplitude da força externa \(F,\) e a frequência de oscilação da força externa \(\omega.\) Para este problema, definimos a função e o macro abaixo, que são usados com a subrotina \verb|solve_rk4|.
\inputminted[firstline=117,lastline=124]{c}{src/main.c}
\subsection{Espaço de fase}
\begin{figure}[!ht]
    \centering
    \includegraphics[width=0.5\textwidth]{src/2a.png}
    \caption{Espaço de fase do potencial poço duplo sem amortecimento ou força externa.}
    \label{fig:ex2a}
\end{figure}
\begin{figure}[!ht]
    \centering
    \includegraphics[width=0.5\textwidth]{src/2b.png}
    \caption{Espaço de fase do potencial poço duplo com amortecimento.}
    \label{fig:ex2b}
\end{figure}
\begin{figure}[!ht]
    \centering
    \includegraphics[width=0.5\textwidth]{src/2c.png}
    \caption{Espaço de fase do potencial poço duplo com amortecimento e força externa.}
    \label{fig:ex2c}
\end{figure}
\subsection{Diagrama de bifurcação}
\begin{figure}[!ht]
    \centering
    \includegraphics[width=0.5\textwidth]{src/2d.png}
    \caption{Diagrama de bifurcação}
    \label{fig:ex2d}
\end{figure}
\subsection{Mapa de Poincaré}
\begin{figure}[!ht]
    \centering
    \includegraphics[width=0.5\textwidth]{src/2e.png}
    \caption{Mapa de Poincaré}
    \label{fig:ex2e}
\end{figure}

\section{Código na íntegra}
A rotina principal, cujas seções foram discutidas anteriormente, se encontra a seguir.
\inputminted[fontsize=\small]{c}{src/main.c}

O programa feito para produzir os gráficos das %\cref{fig:ex1,fig:ex2,fig:ex3} 
se encontra abaixo.
\inputminted[fontsize=\small]{python}{src/plot.py}
\end{document}
